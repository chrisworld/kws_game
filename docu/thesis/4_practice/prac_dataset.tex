% --
% dataset

\section{Dataset}
Two dataset are used for training, one is the Speech Commands Dataset from \cite{warden2018} and one is self made, denoted here as \enquote{My Dataset} consisting only of 5 labels.
All Datasets merely consists of raw wav files, no feature extraction was done beforehand.
This is very convenient, since the feature extraction is completely up to the users.
It is mentioned, because some datasets provide the feature extraction for their users so that only the Neural Network Architectures are of importance.
However it is now very important how features are extracted and this might differ from many users, such that a direct comparance of scores between two independent researchers using different methods in feature extraction is quite difficult, in regards of Neural Network performances.

Some abbreviations and references were done, so that the jungle of selected parameters get a little bit more clear to the reader of this thesis.
The abbreviations of the dataset are shown in \rtab{dataset_abbr}.

The Speech Commands Dataset is extracted before it is used for training. 
To reduce computations in the evaluation process of Neural Networks, it was important to reduce the number of classes and examples per class to an suitable number.
%In the Evaluation of Neural Networks the datasets the references in \rtab{dataset_abbr} are used

%\begin{table}[ht!]
\begin{center}
\caption{Dataset references with label restrictions in number of labels and labels itself.}
\begin{tabular}{ M{2cm}  M{2cm}  M{5cm} }
\toprule
%\multicolumn{4}{c}{\textbf{Feature Groups}} & \multicolumn{2}{c}{\textbf{Accuracy}} \\
\textbf{Reference name} & \textbf{Number of examples per label} & \textbf{Selected Labels}\\
\midrule
n500-c5 & 500 & left, right, up, down, go\\
n500-c10 & 500 & yes, no, left, go, down, off, right, stop, up, on\\
n500-c30 & 500 & \enquote{all}\\
\bottomrule
\label{tab:dataset_labels}
\end{tabular}
\end{center}
\end{table}
\FloatBarrier
\noindent

%\begin{table}[ht!]
\begin{center}
\caption{Dataset reference with feature group extraction.}
\begin{tabular}{ M{2cm} M{8cm} }
\toprule
%\multicolumn{4}{c}{\textbf{Feature Groups}} & \multicolumn{2}{c}{\textbf{Accuracy}} \\
\textbf{Abbreviation} & \textbf{Meaning}\\
\midrule
c[0-1] & use of cepstral features, 0 is false and 1 is true\\ 
d[0-1] & use of delta features\\ 
dd[0-1] & use of double delta features\\ 
e[0-1] & use of energy features\\
norm & features are normalized over frames\\
\bottomrule
\label{tab:feature_groups}
\end{tabular}
\end{center}
\end{table}
\FloatBarrier
\noindent

\begin{table}[ht!]
\begin{center}
\caption{Dataset abbreviations for label selection and feature group extraction.}
\begin{tabular}{ M{2cm} M{9cm} }
\toprule
%\multicolumn{4}{c}{\textbf{Feature Groups}} & \multicolumn{2}{c}{\textbf{Accuracy}} \\
\textbf{Abbreviation} & \textbf{Meaning}\\
\midrule
L5 & Selected labels: left, right, up, down, go\\
L10 & Selected labels: yes, no, left, go, down, off, right, stop, up, on\\
L30 & Selected labels: \enquote{all}\\
n[0-9]+ & Number of examples per class label, e.g. n500\\
\midrule
c[0-1] & Feature Group, use of cepstral features, 0 is false and 1 is true\\ 
d[0-1] & Feature Group, use of delta features, \ditto\\ 
dd[0-1] & Feature Group, use of double delta features, \ditto\\ 
e[0-1] & Feature Group, use of energy features, \ditto\\
norm[0-1] & features are normalized over frames\\
\bottomrule
\label{tab:dataset_abbr}
\end{tabular}
\end{center}
\end{table}
\FloatBarrier
\noindent

