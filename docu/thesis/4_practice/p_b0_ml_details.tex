% --
% ml details

\subsection{Training Details}
We can separate the training details into the categories:
\begin{enumerate}
  \item Dataset parameters
  \item Features extraction parameters
  \item Feature selection
  \item Transfer Learning parameters
  \item Machine Learning Training parameters
\end{enumerate}
The dataset parameters are the information of what labels from the dataset are used and how many examples per labels.
The feature extraction parameters simply give information about how features are extracted, e.g. this includes the hop size, frame size, filter bands of the MFCC, etc.
The feature selection is the information about what input feature groups are used in the training, e.g. use cepstral coefficients only, or add delta and energie features.
The Transfer Learning parameters are pre-trained weights for the actual neural network architecture to be trained. 
This could be only the first convolutional layers or entire networks. 
In this thesis only the convolutional layers are transfered from different learning methods.
The Machine Learning training parameters are classicaly parameters such as learning rate, number of epochs, etc.

\begin{table}[ht!]
\begin{center}
\caption{Dataset Details and References}
\begin{tabular}{ M{2cm}  M{2cm}  M{5cm} }
\toprule
%\multicolumn{4}{c}{\textbf{Feature Groups}} & \multicolumn{2}{c}{\textbf{Accuracy}} \\
\textbf{Reference name} & \textbf{Number of examples per label} & \textbf{Selected Labels}\\
\midrule
500-c5 & 500 & left, right, up, down, go\\
\bottomrule
\end{tabular}
\end{center}
\label{tab:ml_details_dataset}
\end{table}
\FloatBarrier
\noindent

