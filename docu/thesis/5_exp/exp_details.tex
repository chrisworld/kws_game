% --
% training details

\section{Implementation and Training Details}\label{sec:exp_details}
\thesisStateNotReady
In this sections, the implementation und parameters for training are is described in more detail.

% --
% implementation notes

\subsection{Implementation notes}\label{sec:exp_details_implementation}
The programming code for this thesis was entirely done in \texttt{Python} with version $>3.8$ evaluated on a linux system.
This might be important if one tries to run the python code on a windows machine, it was not tested for it and could yield in errors (especially regarding paths variables).
For the neural networks implementation and training, the framework \texttt{Pytorch} with version $1.7.0$ was used. 
Usually it should not be a problem if a newer version of \texttt{Pytorch} is applied.
The feature extraction with Mel Frequency Cepstral Coefficients (MFCC) was done with an own implementation, but using efficients functions for transforms functions, such as the short time fourier transform, with packages from \texttt{scipy}.
Matrix computations usually are done with the package \texttt{numpy}.
Several other \texttt{Python} packages were used within the project, but are not named explicitly.


% --
% training details

\subsection{Neural Network Training Details}\label{sec:exp_details_training}
The training details of the used neural networks can be split into following parameters:
\begin{enumerate}
  \item Features extraction parameters
  \item Dataset parameters
  \item Feature selection
  \item Transfer Learning parameters
  \item Training parameters
\end{enumerate}
The feature extraction parameters provide information about how the MFCC feautes were extracted.
Note that no
During the feature selection experiments the parameters for cepstral coefficients with enhancements are varying normalization.
If not other stated, the feature extraction parameters in \rtab{exp_details_params_feature} are used.
% --
% feature extraction parameters
\begin{table}[ht!]
\begin{center}
\caption{Parameters for MFCC feature extraction.}
\begin{tabular}{ M{6cm}  M{2cm} M{2cm}}
\toprule
\textbf{Parameter} & \textbf{Value} & \textbf{Varying for experiments} \\
\midrule
Length of signal & \SI{500}{\milli\second} & - \\
Analytic window size & \SI{25}{\milli\second} & -\\
Hop size & \SI{10}{\milli\second} & -\\
Window Function & Hanning & -\\
\midrule
Number of filter bands & 32 & -\\
Number of cepstral coefficients & 12 & yes\\
Delta features & 12 & yes \\
Double delta features & 12 & yes \\
Energy features & 3 & yes \\
Frame based normalization & yes & yes\\
\bottomrule
\label{tab:exp_details_params_feature}
\end{tabular}
\end{center}
\end{table}
\FloatBarrier
\noindent
The dataset parameters are the selected labels and the number of examples per labels.
The selected labels are either the 12 labels for comparison to the benchmark networks described in \rsec{prev_kws_benchmark} or 7 labels used for the deployed KWS Game are listed as well as the number of examples per label in \rtab{exp_details_params_dataset}.
% --
% dataset parameters
\begin{table}[ht!]
\begin{center}
\caption{Parameters for the dataset extraction.}
\begin{tabular}{ M{7cm}  M{6cm}}
\toprule
\textbf{Parameter} & \textbf{Value} \\
\midrule
class dictionary with 7 labels (L7) & \{\enquote{left},  \enquote{right}, \enquote{up}, \enquote{down}, \enquote{go}\}\\
class dictionary with 12 labels (L12) & \{\enquote{left},  \enquote{right}, \enquote{up}, \enquote{down}, \enquote{go}\, \enquote{stop}, \enquote{yes}, \enquote{no}, \enquote{on}, \enquote{off}\}\\
\midrule
Number of examples per label & 500 \\ 
\bottomrule
\label{tab:exp_details_params_dataset}
\end{tabular}
\end{center}
\end{table}
\FloatBarrier
\noindent
%The Abbreviation regarding dataset parameters and feature selection were already listed in \rtab{exp_dataset_abbr}.
%The feature selection is the information about what input feature groups are used in the training, e.g. use cepstral coefficients only, or add delta and energy features, their references are shown in \rtab{dataset_feature_groups}.
The transfer learning parameters describe the training details of the adversarial networks and what
For instance it describes if the weights are used from the discriminator or generator network or how much epochs were used for the adversarial training, etc.
%The Abbreviations for training parameters can be specified as listed in \rtab{exp_details_adv}
%\begin{table}[ht!]
\begin{center}
\caption{Adversarial Training abbreviations.}
\begin{tabular}{ M{2cm}  M{5cm} }
\toprule
%\multicolumn{4}{c}{\textbf{Feature Groups}} & \multicolumn{2}{c}{\textbf{Accuracy}} \\
\textbf{Abbreviations} & \textbf{Meaning}\\
\midrule
dec & use of decoder weights\\
enc & use of encoder weights\\
itl[0-9]+ & iterations per label, e.g. itl500 for 500 iterations\\
\bottomrule
\label{tab:exp_details_adv}
\end{tabular}
\end{center}
\end{table}
\FloatBarrier
\noindent


The training parameters are classical parameters for neural network training, such as learning rate, number of epochs, etc.
%Their selection and references are listed in \rtab{exp_details_train_params}
%\begin{table}[ht!]
\begin{center}
\caption{All training parameters used within this thesis and their abbreviations.}
\begin{tabular}{ M{2cm}  M{5cm} }
\toprule
%\multicolumn{4}{c}{\textbf{Feature Groups}} & \multicolumn{2}{c}{\textbf{Accuracy}} \\
\textbf{Abbreviations} & \textbf{Meaning}\\
\midrule
it[0-9]+ & Number of epochs (or iterations)\\
bs[0-9]+ & Batch size, e.g. bs32 is a batch size of 32 examples\\
lr[0-9.]+ & Learning rate, e.g. lr0.0001\\
mo[0-9.]+ & Momentum, e.g. mo0.5\\
\bottomrule
\label{tab:exp_details_train_params}
\end{tabular}
\end{center}
\end{table}
\FloatBarrier
\noindent

