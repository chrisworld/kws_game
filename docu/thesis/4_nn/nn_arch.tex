% --
% Neural Network Architectures

\section{Neural Network Architectures}\label{sec:nn_arch}
All neural network Architectures evaluated within this thesis are presented here.
In general the used architectures were:
\begin{enumerate}
	\item Convolutional Neural Networks (CNN)
	\item Generative Adversarial Neural Networks (GAN)
	\item Wavenets
\end{enumerate}
The term convolutional neural network here consists of all architectures consisting of at least one convolutional layer and the intention to simply classify each speech commands from each other. 
Therefore the output of a convolutional net is of size of the number of individual speech commands and usually has some kind of probability distribution or energy equivalence.

With adversarial neural networks all architectures are meant, with at least two separate neural network Architectures, e.g. a Discriminator and a Generator Network and the intention to outperform the other Network in a task where both play a game against each other.
The word game is meant in the sense of game theory, where the goal is to find an equilibrium state of players competing against each other.
An equilibrium state is usually found if all players are satisfied with the outcome.
In this thesis the amount of players, or neural networks, is always two.
An overview of all models is shown in \rtab{nn_arch_overview} with abbreviations in \rtab{nn_arch_abbreviation}.
\begin{table}[ht!]
\begin{center}
\caption{Network Architectures Abbreviations}
\begin{tabular}{ M{2.5cm}  M{10cm} }
\toprule
\textbf{Abbreviations} & \textbf{Meaning}\\
\midrule
c[0-9] & convolutional layer with layer number\\
f[0-9] & feed forward fully connected layer with layer number\\
m[0-9] & max pooling layer layer with layer number\\
ch & input channel number for mfccs it is usually 1\\
fs & frame size (usually 50 -> 50ms)\\
ms & feature size (mfcc), depends on feature selection\\
cf & output number of last flattened convolutional layer\\
cl & number of class labels\\
\bottomrule
\label{tab:nn_arch_abbreviation}
\end{tabular}
\end{center}
\end{table}
\FloatBarrier
\noindent
\begin{table}[ht!]
\begin{center}
\caption{Network Architectures Overview with reference names}
\begin{tabular}{ M{2.5cm}  M{2.1cm}  M{2.1cm} M{2.1cm} M{2.5cm}}
\toprule
%\multicolumn{4}{c}{\textbf{Feature Groups}} & \multicolumn{2}{c}{\textbf{Accuracy}} \\
\textbf{Reference name} & \textbf{Feature maps} & \textbf{Kernel sizes} & \textbf{Strides} & \textbf{Feed Forward} \\
\midrule
conv-trad & c1: (ch, 64) c2: (64, 64) & c1: (4, 20) mp: (2, 4) c2: (2, 4) & c1: (1, 1) mp: (2, 4) c2: (1, 1) & f1: (cf, 32) \quad f2: (32, 128) f3: (128, cl)\\
\midrule
conv-fstride & c1: (ch, 54) & c1: (8, fs) & c1: (4, 1) & f1: (cf, 32) \quad f2: (32, 128) \quad f3: (128, 128) \quad f4: (128, cl)\\
\midrule
conv-encoder-fc1 & c1: (ch, 48) \quad c2: (48, 8) & c1: (ms, 20) \quad c2: (1, 5) & c1: (1, 1) \quad c2: (1, 1) & f1: (cf, cl)\\
\midrule
conv-encoder-fc3 & c1: (ch, 48) \quad c2: (48, 8) & c1: (ms, 20) \quad c2: (1, 5) & c1: (1, 1) \quad c2: (1, 1) & f1: (cf, 64) \quad f2: (64, 32) \quad f3: (32, l)\\
\bottomrule
\label{tab:nn_arch_overview}
\end{tabular}
\end{center}
\end{table}
\FloatBarrier
\noindent

