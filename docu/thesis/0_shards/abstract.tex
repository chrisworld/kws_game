% --
% abstract

Key Word Spotting is a valuable tool in the interaction between humans and machines in various situations.
This thesis is specialised on the domain of Video Games and evaluates possible neural network architectures for speech command classifications tasks.
In focus are Convolutional Neural Network (CNN) architectures with adversarial pre-training using Mel Frequency Cepstral Coefficients (MFCC) as input features and Wavenets applied on raw audio samples.
The energy consumption and efficients is important in video games and therefore a strong criterion within this thesis.
Further some possible video game ideas or scenarios with speech command inputs with the purpose of triggering events within the games are evaluated and discussed.