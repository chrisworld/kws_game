% --
% prev key word spotting

\section{Works on Key Word Spotting with Neural Networks}\label{sec:prev_kws}
Some works regarding KWS with neural networks are presented here. 
The aspect of energy efficients is very important within this thesis, as the KWS system is deployed to a video game and has to work in real-time.
Further the neural network architectures evaluated in this thesis use an already profoundly evaluated speech command dataset from \cite{Warden2018} with many different solution concepts on neural networks. 
A benchmark is therefore given on classification accuracies.

% energy efficient
\subsection{Energy efficient solutions}


% benchmark
\subsection{Benchmark Networks for this thesis}\label{sec:prev_kws_benchmark}
First it is to mention that the speech command dataset \cite{Warden2018} consists of raw audio data and there is no pre-processing or separation in any evaluation, test or train set. 
Therefore its already difficult to compare accuracies between the works, as the feature extraction and quality of the selected samples for the test set has some impact.
Also some works include noise samples or create an own class for a mixed word label.
Further the speech command dataset uses two versions \texttt{v1} and \texttt{v2}, where \texttt{v1} has 30 classes and \texttt{v2} 10 classes of speech commands.
Still one can rely on the accuracies obtained from the papers and usually everything that is over $90\%$ accuracies for 10 classes and over $80\%$ accuracies for 30 classes is quite good.
A good overview of actual benchmark scores regarding the speech command dataset is given in \cite{PaperswithcodeKWS}.



