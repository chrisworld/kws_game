% --
% video games with speech commands

\section{Video Games with Speech Input}
Video games with speech inputs are an rarely occurrence in the gaming industry, though it is a very interesting and immersive way to interact.
Technically the voice of the player of the video game, has to be recorded by a microphone.
The input stream from this microphone is then processed through a online or realtime system to convert the intention of the player to a real action within the game.
Easy to define, but not that hard to implement compared to other more hardware based input channels, such as a click on a keyboard button.
Also speech input is very slow compared to hardware input, where players interact within tens of milliseconds (the input lag of gaming controllers should ideally be under 50ms).
This is not possible with speech, the player has to physically form a waveform, representing the action in the game.
Further the waveform has to be pre-processed, feautre extracted and classified to the best estimation on the available key words.
Concluding this, a good estimate to create and process a speech input is about under one second, the less the better for the playing experience of the players.

Maybe those issues and the complexity of the task hinders game developers to produce more content with Key Word Spotting.


% A speech input for video games is a spoken waveform, recorded through a microphone, from any speaker intending to inflict a certain change while playing a Video Game. 
% Certainly this spoken waveform has to be processed, such as other inputs channels have to be (like keyboard buttons pressed), so that its meaning can be understood by the computer system behind the game. 
% Although this processing of a waveform is much more complicated and prone to errors, compared to a simple click on the keyboard or mouse. 
% That might be one of the reasons why Speech Inputs are very rare to be found in Video Games, still they exist.