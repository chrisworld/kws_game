% --
% Introduction

\chapter{Introduction}\label{sec:intro}
Key Word Spotting (KWS) is the task of identifying or classifying spoken words from human speakers out of a limited vocabulary or set of key words. 
For example it is the task of identify the spoken word \enquote{left} out of the vocabulary \{\enquote{left}, \enquote{right}\}. 
This problem is solved by humans everyday with not much effort, but it is a real challenge for computer systems.

The task of KWS is not a new one, it is solved with a variety of different approaches.
One approach to solve it on a computer system, is to use the advances in Machine Learning, or more specifically in this thesis, the advances in Neural Networks.

KWS systems are since long ago no science-fiction anymore. 
There already exists many real world applications for consumers in everyday situations.
Slowly Key Word Spotting, or generally speaking speech recognition, become accepted and widely used in the society of human beings.
%(e.g. \textit{Alexa} from the company \textit{Amazon}).

Also video games are a potential application to KWS, for instance to control some elements within the game.
Until now there not that many game out there with the ability to be controlled by their players voice. 
It is the aim to get some light into the shadows of the reasons, why it did not become more popular.


In summary, the focus of this thesis lies on Key Word Spotting with Neural Networks in supervised learning on a Speech Command Dataset \cite{Warden2018}.
Further the best suitable solutions of speech commands classification for video games should be presented and evaluated.

%However unfortunately due to the (usually) large amount of data and complex Neural Network Architectures it is in most cases not tractable, how these systems learn to distinguish between different samples. Nevertheless the technical benefits and their performance made them a standard tool in modern Machine Learning tasks.

% general intro
% --
% Intro of key word spotting

\section{The task of Key Word Spotting}\label{sec:intro_kws}
As described earlier Key Word Spotting (KWS) is the task of classifying a spoken word as key word out of a set of key words.
The set of key words or vocabulary is denoted as:
% kws dict
\begin{equation}\label{eq:intro_kws_dict}
	S \coloneqq \{s_i, \quad i=1 \dots c\}
\end{equation}
with a total number of $c$ key words denoted individually as $s_i$.
The task is to select the key word closest to the spoken key word from the user, denoted as target $t$.
The target does not necessarily being a member in the set of key words $S$ in fact it can be any arbitrary word.
With the abstract formulation:
% kws task
\begin{equation}\label{eq:intro_kws_task}
	\hat{s} = \underset{s_i \in S}{\arg \min} \, \mathcal{D}(t, s_i)
\end{equation}
the most appropriate key word $\hat{s}$ can be selected, where $\mathcal{D}$ is some kind of distance measure between two words.
%Further the definition of the distance measure $\mathcal{D}$ is hard to define.
The formulation in \req{intro_kws_task} is merely semantic, but KWS in computer systems must cope with various transformations of raw input samples of audio data denoted as $x \in \R^n$, with a total number of $n$ samples.
%From the audio data an inference to an output class labels $y \in \R^c$, through for instance a neural network, representing the key words individual probabilities or energy value of similarity, the most probable is picked by:
From the audio data an inference to output class probabilities $y \in \R^c$, through for instance a neural network with softmax output, the most probable is picked by:
\begin{equation}\label{eq:intro_kws_class}
	\hat{s} = \{s_i \, | \, \underset{i = 1 \dots c}{\arg \max} \, y_i\}
\end{equation}
if the highest value of $y_i$ for all $i$ is the most probable index or indices for a spotted key word or key words.

In comparison to full Automatic Speech Recognition (ASR), where whole sentences are identified, key word spotting operates merely on the word level.
Therefore KWS is a bit easier to deploy and less complex than ASR.
On the other hand KWS systems, used in practical application, must run very energy efficiently on low energy devices, such as mobile phones, and give immediate and accurate responses to the users. 
A good elaboration on the requirements of KWS systems can be found in the motivation section in \cite{Warden2018}.
%Concluding this KWS systems can be very difficult, however with neural networks and available data resources, a tool emerged to solve this complex problem without thinking too much about all its technical details.


% --
% Intro to neural networks

\section{Neural Networks for Key Word Spotting}\label{sec:intro_nn}
\thesisStateReady
Neural networks enable computers to automatically learn from data to be able to solve tasks such as pattern recognition in images or audio.
The examples or samples from the input data can be paired with annotations, denoted as \emph{labels} or \emph{classes}.
If the label information of each example is used during the \emph{training} of a machine learning system, such as a neural network, it is called \emph{supervised learning} otherwise it is called \emph{unsupervised learning}, however supervised learning is more commonly applied.

%The big advantage of neural networks is that they are able to cope with huge amounts of input variables per data example and are able to extract their own features of those inputs through many layers within the network.
The big advantage of neural networks is that they are able to cope with large amounts of input variables per data example.
Considering a raw waveform file of merely \SI{1}{s} time duration, sampled with \SI{16}{\kilo\hertz} would give a input size of 16000 features.
This huge amount of input features is even difficult for neural networks to learn from and usually a feature extraction stage is placed in between to reduce the input dimension.
For instance the computation of MFCCs, using 12 out of 32 coefficients and a time duration of \SI{0.5}{s} with a time shift of \SI{10}{\milli\second}, reduces the input feature size dramatically to $12 \times 50 = 600$, which is still a high number of input features, but much more affordable and faster to train.

%In this thesis, the word \emph{feature} has several meanings, one is as name of extracted data and therefore be the same as input variables. 
%Another is, that a feature is simply some kind of compressed representation of a high dimensional data.
Neural networks are able to learn own feature representations, selection and interpretation, rather than using hand-crafted ones done by humans with expertise in the application.
Note that hand-crafting features of a complex recognition task, is in most cases not even possible or extremely cumbersome.
So researchers prefer neural networks because of their easy deployment scheme and state of the art performances.
Further it enables everyone who is capable of using neural network tools, to create solution to rather complex problems usually solved by experts in the field, given there is enough data and processing power available.
Therefore elaborate feature extraction stages become less important to the users.
This on the other side may lead into less understanding of the actual problem and more \enquote{try and error} approaches of different neural network architectures and training parameters.
The energy consumption required to train large neural network with many parameters on a huge training dataset, shall not be forgotten, especially in times of climatic change.
Reusing pre-trained weights from renowned network architectures is a good way to reduce energy consumption in finding an optimal classifier for a specific task.
The re-usability of pre-trained weights is often named as \emph{transfer learning}. 
A small summary on transfer learning can be found in \cite{TransferLearning}.
%This led to the thinking that everyone, who owns data and computational power, is the superior of solving complex problems such as image or audio classifications.
%However it is not always like this rather negative example, when using neural network approaches.

The potential of neural networks in research are vast.
The observation on how the learning from data is done and what recognition patterns are obtained after training, might allow researchers and experts to better understand the problem or gain a different viewpoint on it.
%However if neural networks are observed on how they are able to learn from data and what recognition patterns they have obtained after training, it might benefit researchers and experts to better understand the problem or gain a different viewpoint on it.
%It is extremely interesting to work with them and get knowlege about how and why they are able to produce such good results.
%This again feedbacks experts to gain more understanding or a different viewpoint on the topic.
Especially when using Convolutional Neural Networks (CNN), researchers are actually able to observe and visualize the learned filters and interpret the results.
A very interesting example of investigating learned CNN filters is shown by Zeiler el. al. \cite{Zeiler2013}.
Other interesting subjects in research are generative models, such as Generative Adversarial Networks (GAN) \cite{Goodfellow2014}, which are able to create convincing samples from the learned data distribution.

Neural network architectures for speech recognition are a little bit different from image recognition, mainly because of the sequential nature of time signals.
However if time signals are restricted in time, that means limited to a fixed number of samples, and frequency features are extracted over that time span, then speech signals can be represented in 2D space (frequency and time) and classified like images.
That suggests that CNNs are reasonable network architectures for speech as well.
Another interesting architecture for audio signals is the Wavenet \cite{Oord2016}, because of its ability to process raw audio data.
%and were originally intended for speech synthesis, but could also be used for recognition tasks.



% --
% video games with speech commands

\section{Video Games with Speech Input}
Video games with speech inputs are an rarely occurrence in the gaming industry, though it is a very interesting and immersive way to interact.
Technically the voice of the player of the video game, has to be recorded by a microphone.
The input stream from this microphone is then processed through a online or realtime system to convert the intention of the player to a real action within the game.
Easy to define, but not that hard to implement compared to other more hardware based input channels, such as a click on a keyboard button.
Also speech input is very slow compared to hardware input, where players interact within tens of milliseconds (the input lag of gaming controllers should ideally be under 50ms).
This is not possible with speech, the player has to physically form a waveform, representing the action in the game.
Further the waveform has to be pre-processed, feautre extracted and classified to the best estimation on the available key words.
Concluding this, a good estimate to create and process a speech input is about under one second, the less the better for the playing experience of the players.

Maybe those issues and the complexity of the task hinders game developers to produce more content with Key Word Spotting.


% A speech input for video games is a spoken waveform, recorded through a microphone, from any speaker intending to inflict a certain change while playing a Video Game. 
% Certainly this spoken waveform has to be processed, such as other inputs channels have to be (like keyboard buttons pressed), so that its meaning can be understood by the computer system behind the game. 
% Although this processing of a waveform is much more complicated and prone to errors, compared to a simple click on the keyboard or mouse. 
% That might be one of the reasons why Speech Inputs are very rare to be found in Video Games, still they exist.
\newpage

% research questions / problem formulation
\section{Problem Formulation for this Thesis}
In this section, several research question regarding Key Word Spotting in Video Games are asked and described in their interpretation.
The problem formulation for this thesis can be split into to 3 parts:

\begin{enumerate}[label={Q.\arabic*)}, leftmargin=1.4cm]
    \item Signal Processing and Feature Extraction of Speech Input.
    \item Neural Network Training and Classification of Speech Commands.
    \item Creating a Game were Speech Commands enhance the game experience.
\end{enumerate}
Note that a Key Word has the same meaning as a Speech Command in this thesis and therefore might be referred in either way.
Further note that not all research questions can be answered within the scope of this thesis in satisfactory way.
Nevertheless those questions can be asked and some solution concepts discussed. 



\subsection{Signal Processing and Feature Extraction Research Questions}
This part focuses on how to acquire a meaningful representation of the input waveform from an human speaker. This representation usually is a feature vector extracted from the raw microphone data of a certain time interval. Further the retrieved feature vector is input to the Neural Network Architecture for classification. Following Questions arise here:

\begin{enumerate}[label={Q.1.\alph*)}, leftmargin=1.75cm]
%    \item Wow is the microphone data processed and when does it show the presence of a speech command.
    \item When should the feature extraction be activated, so to reduce computations?
    
    \label{it:q1-a}
    
    \item Which time interval should be captured to represent a speech command?
    \label{it:q1-b}
    
    \item Does the signal processing have to be invariant to background noise and especially to game sounds?
    \label{it:q1-c}
    
    \item What are meaningful features for speech recognition?
    \label{it:q1-d}
    
\end{enumerate}
\noindent
\textbf{Question \ref{it:q1-a}:} 
It is crucial to reduce computations in a running game, so that the game is not slowed down with unnecessary processing of meaningless input data.
%as it is not meaningful to compute features the whole time. 
Ideally a feature vector is only processed, when there is actually a speech command present from a human speaker. 
This however is not always trivial.
To indicate that a speech command is present, one possibility is to compute the, relatively efficient calculation, of an energy value within a certain time interval of the raw input data and have a simple threshold value decide, when a speech command is available. 
The downsides of this approach is, that the microphone and the background sound (including the game sound) should be less energy intensive than the speech command of the speaker, so that a speech command is not triggered the whole time.
%Therefore the microphone should be close to the speaker and capture more of the speech commands and less of the background and gaming sound.
%Also it might happen that some disturbance of the microphone, e.g. mechanical strike to the mic, yield an actual command.
Another approach would be to indicate a speech command with a e.g. say a click of a certain button on the keyboard, and use the push and talk principle. 
These methods with more details shall be discussed in further sections.

\textbf{Question \ref{it:q1-b}:} 
The restriction to processing input data to a feature vector in a certain time interval is essential for the design of the Neural Network.
But more importantly it is the restriction of how long a human speaker has time to speak a speech command so that the whole command is captured. If a human speaker prolongs the pronunciation of a word, e.g. \enquote{left} for lets say 1 second, hardly all is captured if the time interval is restricted to say 500 milliseconds. If this 500 milliseconds is then sufficient for a still correct classification, must be evaluated. 
Also in the application of a game, the user should speak commands with short duration, so that the game reacts fast. Another downside would be if one repeatedly speaks a speech commands, so that the time interval of another command would overlap each other. Ideally the time interval is flexible, but this is harder to implement than a fixed time interval.

\textbf{Question \ref{it:q1-c}:}
Usually low background noise should not be a problem for Neural Networks trained on a large enough data set. 
A more difficult problem are the game sounds when turned up loud enough and without use of headphones during playing. 
Therefore the microphone will not only capture the voice of a speaker, but also a fair amount of unwanted game sounds. 
This problem seems to be theoretically solvable, as the shape of the nuisance is known and could therefore be cut out in some way. 
However in practise this might be hard to solve, so that the signal of interest is not disturbed. 
A solution to this problem would probably take too much time and should be prioritized low in this thesis. 
However playing a video game without game sound is unsatisfying and therefore this problem should be solved in future works.

\textbf{Question \ref{it:q1-d}:} 
Meaningful features for speech is a classical problem in speech recognition.
Therefore it is important to know what a Word essential is composed of. A Word is a sequential combination of either vowels (e.g. a, e, ...) or consonants (e.g. k, l, ...) with a certain length. In linguistics for instance, one can distinguish vowels with frequency peaks in a spectogram of this vowel, where a spectogram is nothing else but a frequency response of small time chunks over a time interval. 
However, due to many different factors involved in speakers, like age, gender, nathionality and physiology of the vocal tract, there is a huge variance in the pronunciation of words from different persons. 
This yields in a difficult problem to solve. 
Usually the Mel Frequency Cepstral Coefficients (MFCC) are used for speech recognition tasks, as they represent frequencies in equidistant mel-bands on the freuqency scale in a spectorgram kind of way and therefore give a good footprint of the speech signal to analyze (more information on MFCC is presented in section ...).

\subsection{Neural Network Implementation Research Questions}
To deploy a Key Word Spotting system with Neural Networks into a game, it is crucial to know the Neural Networks Architecture and its in- and output representations. 
While the inputs are the features from the signal processing, the outputs are the classes representing the speech commands of the game. 
An inference is then done by the neural network, which is the process of classification of an input feature to one output class.
The Architecture of an neural network describes how the features are processed through layers in the network.
Further every neural network has to train its parameters with enough data and some special techniques so to generalize well to unseen data.
Following Questions can therefore be asked here in general:

\begin{enumerate}[label={Q.2.\alph*)}, leftmargin=1.75cm]
    \item What vocabulary of speech commands is used in the game and is there enough training data with sufficient diversity available for the neural network to learn from?
    \label{it:q2-a}
    
    \item What happens if an input feature represents a spoken word, which is not in the speech commands vocabulary (Non Key Word) and how should this exception be handled?
    \label{it:q2-b}
    
    \item What is the best neural network architecture so that the classification yields good results and the game is not slowed down during the inference process.
    \label{it:q2-c}
    \begin{enumerate}[label=(\roman*)]
        \item Are Wavenets a solution to this task? 
        \item Can Adversarial Networks improve the generalization?
    \end{enumerate}
    
\end{enumerate}
\noindent
\textbf{Question \ref{it:q2-a}:} The question of availability of a speech command data set can be answered right away, as there exists with enough and diverse data (more about the dataset in section ...). The more important question is, which of these speech commands should be used for the game? This mainly depends on the game itself and the actions to be controlled. Usually commands like \enquote{left}, \enquote{right}, etc. are a good choice to move things within a game for instance.
Another point is to restrict the amount of speech commands, so that a simpler neural network architecture can be deployed, which of course should still be sufficient for a good classification rate.

\textbf{Question \ref{it:q2-b}:} Without doubt players will try out words, which are not in the speech commands vocabulary (denoted as Non Key Words) and observe what happens.
The ideal response would be that nothing happens or an indication is shown that the word is not present in the vocabulary. 
However it might happen that the similarly of a Non Key Word is too close to a Key Word, so that a command is triggered in the game. 
At the same time the neural network should not classify Key Words as Non Key Words, which is even more important, that the game is not interrupted or disturbed.
It is better to rely, that players are using Key Words most of the time, so that they are preferred over Non Key Words.

\textbf{Question \ref{it:q2-c}:}
Several different neural network approaches with a low computational footprint should be tested and compared with each other regarding classification rate and energy efficiency. 
A video game with online speech input restricts the amount of computation and time for classification by the minimum frames per second (FPS) a game should be played.
This is because the FPS should not fall under a certain limit (usually 30 FPS in video games), otherwise the fluidity of the game is not ensured.
Further Wavenets and Adversarial Networks shall be evaluated regarding their value in this task.
\subsection{Video Games with Speech Commands Research Questions}
Video Games using speech commands as inputs are a very rarely seen curiosity in the gaming industry and therefore it is important to show and discuss its capability. This rather empirical section asks following important question:

\begin{enumerate}[label={Q.3.\alph*)}, leftmargin=1.75cm]
    \item What is the added value of speech commands in the gaming experience of players and what do game developers need to consider, when designing a game with speech commands?
    \label{it:q3-a}
    
\end{enumerate}
\noindent
\textbf{Question \ref{it:q3-a}:} The Game Design is the focus of this question and it should be answered in the player and game developers view.
In certain video game scenarios, speech commands are very useful, interesting and enhance the gaming experience, in other they might even disturb the game play or spoil it completely.
It certainly can be stated, that speech commands are not always reliable and therefore a main game mechanic solely based on speech commands is not preferable.
Therefore a game developer has to design a game with speech commands with care.
Some game ideas and prototypes should be shown and some existing games discussed.

%\textbf{Question \ref{it:q3-b}} is about the problems and obstacles a game developer meets, when deciding to use speech commands. It is about how

\section{Visual Guidance}
To get a better overview on the presentation of data and results, context specific color color-schemes are used within this thesis.
% There exist following context abstractions:

% \begin{itemize}
%     \item raw waveforms from soundfiles
%     \item extracted features, e.g. MFCCs
%     \item weights matrices of neural network models
%     \item training scores
% \end{itemize}




