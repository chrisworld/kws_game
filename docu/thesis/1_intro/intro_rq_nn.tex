\subsection{Neural Network Implementation Research Questions}
To deploy a Key Word Spotting system with Neural Networks into a game, it is crucial to know the Neural Networks Architecture and its in- and output representations. 
While the inputs are the features from the signal processing, the outputs are the classes representing the speech commands of the game. 
An inference is then done by the Neural Network, which is the process of classification of an input feature to one output class.
The Architecture of an Neural Network describes how the features are processed through layers in the network.
Further every Neural Network has to train its parameters with enough data and some special techniques so to generalize well to unseen data.
Following Questions can therefore be asked here in general:

\begin{enumerate}[label={Q.2.\alph*)}, leftmargin=1.75cm]
    \item What vocabulary of speech commands is used in the game and is there enough training data with sufficient diversity available for the Neural Network to learn from?
    \label{it:q2-a}
    
    \item What happens if an input feature represents a spoken word, which is not in the speech commands vocabulary (Non Key Word) and how should this exception be handled?
    \label{it:q2-b}
    
    \item What is the best Neural Network Architecture so that the classification yields good results and the game is not slowed down during the inference process.
    \label{it:q2-c}
    \begin{enumerate}[label=(\roman*)]
        \item Are Wavenets a solution to this task? 
        \item Can Adversarial Networks improve the generalization?
    \end{enumerate}
    
\end{enumerate}
\noindent
\textbf{Question \ref{it:q2-a}:} The question of availability of a speech command data set can be answered right away, as there exists with enough and diverse data (more about the dataset in section ...). The more important question is, which of these speech commands should be used for the game? This mainly depends on the game itself and the actions to be controlled. Usually commands like \enquote{left}, \enquote{right}, etc. are a good choice to move things within a game for instance.
Another point is to restrict the amount of speech commands, so that a simpler Neural Network Architecture can be deployed, which of course should still be sufficient for a good classification rate.

\textbf{Question \ref{it:q2-b}:} Without doubt players will try out words, which are not in the speech commands vocabulary (denoted as Non Key Words) and observe what happens.
The ideal response would be that nothing happens or an indication is shown that the word is not present in the vocabulary. 
However it might happen that the similarly of a Non Key Word is too close to a Key Word, so that a command is triggered in the game. 
At the same time the Neural Network should not classify Key Words as Non Key Words, which is even more important, that the game is not interrupted or disturbed.
It is better to rely, that players are using Key Words most of the time, so that they are preferred over Non Key Words.

\textbf{Question \ref{it:q2-c}:}
Several different Neural Network approaches with a low computational footprint should be tested and compared with each other regarding classification rate and energy efficiency. 
A video game with online speech input restricts the amount of computation and time for classification by the minimum frames per second (FPS) a game should be played.
This is because the FPS should not fall under a certain limit (usually 30 FPS in video games), otherwise the fluidity of the game is not ensured.
Further Wavenets and Adversarial Networks shall be evaluated regarding their value in this task.